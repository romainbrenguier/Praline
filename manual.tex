\documentclass{article}
\def\praline{\texttt{PRALINE}}
\title{\praline~Users Manual}
\date{}

\begin{document}
\maketitle
\section{Compiling}
To build \praline, ocaml, ocamlbuild and ocamlgraph are needed.
Once you extracted the files from the archive, the following command:
\begin{verbatim}
make main.native
\end{verbatim}
should build the main executable of the tool.

To edit game graph graphically you can consider using an editor for
the GML language (see
{http://en.wikipedia.org/wiki/Graph\_Modelling\_Language}).

\section{Basic Usage}
The basic way to use the tool is to give it a game file:
\begin{verbatim}
./main.native examples/example1.game
\end{verbatim}
This should write the solution for the game in separate files.

\section{Input Files}

\subsection{The Arena}
The arena as to be given in the dot or gml file format.
There is no constraints on the labels of the states, but edges have to
be labeled by actions separated by a comma (\texttt{,}), actions are
strings (avoid \texttt{\"} and \texttt{,} symbols), there is one action
by player on every edge.
Here is an example of edge in the graphviz format:
\begin{verbatim}
  "3" -> "5" [label="n,y,w"];
\end{verbatim}
Here an example in the GML format:
\begin{verbatim}
  edge [ source 1 target 2 label "1,0,0" ]
\end{verbatim}


\subsection{The Game File}
In the game file, there as to be the path to the file containing the
arena, either in the dot or gml file format. This path should be given
in parentheses after the keyword \texttt{arena}.
\begin{verbatim}
arena "example1.gml"
\end{verbatim}

Optionnaly a starting configuration can be specified, only the
solutions that start in this state will be kept.
\begin{verbatim}
start "2"
\end{verbatim}

Then the objectives for the players are given. 
After the keyword \texttt{objective}, there is the number of the
player (starting at 1), then the type of objective: \texttt{reach} for
reachability, \texttt{safety} for safety and \texttt{buchi} for
B\"uchi objectives, then the target states.
Optinally for reachability and B\"uchi objectives this target states
can be attributed different payoffs.
For example:
\begin{verbatim}
objective 1 safety 6
objective 2 reach 5 -> 2 ; 4 -> 1
objective 3 buchi 4 -> 2 ; 6,5 -> 1
\end{verbatim}

\section{Options}

\paragraph{Displaying the payoff:}
The option \texttt{-dp} display the payoff of all the solutions that
can be found with the tool.

\paragraph{Displaying informations:}
The option \texttt{-i} displays some informations about the game, like
the number of states and edges.

\paragraph{Shapes of the solutions:}
The options \texttt{-os} and \texttt{-og} allow to output a graph
file containing the shapes of the solutions, the first one in the dot
file format and the second in the GML file format.

\paragraph{Computing the suspects:}
\texttt{-susp}

\paragraph{Constructing the strategies:}
\texttt{-ost}

\paragraph{Constructing the product:}
\texttt{-op}

\paragraph{Displaying the objectives:}
\texttt{-do}

\paragraph{Computing the repellors:}
\texttt{-or}

\paragraph{Constraints on the payoff:}
\texttt{-c player min\_payoff} add a constraint on the payoff

\end{document}
